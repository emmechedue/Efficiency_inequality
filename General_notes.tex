\documentclass{article}
\usepackage{graphicx}
\usepackage{amssymb,amsmath}

\author{Stefano Duca}
\title{General notes on the model}

\begin{document}

\maketitle

\section*{Notes on recent changes}

\emph{Madis 04.06.2015 20.56}:

I made two different directories \emph{Single} and \emph{Ensemble} for the single game and ensemble game simulations. Due to the lack of a better word I'll be naming one "single game" in the ensemble game as an \emph{society}. 
 

\section*{Notes on the code and definition of functions}
\textbf{General notes on the functioning of the program:}

\begin{itemize}

\item In the function I have to keep renormalizing the wealth all the time because otherwise the number were too big or too small and the computer was going nuts.

To still print the efficiency I have to compute it at every time step. Right after that I renormalize the wealth array.

To take into account the fact that I might do many iterations before I actually print I have to keep track of how many times I am iterating without printing and update the average of the efficiency.

Hence with this program I'm measuring the efficiency as increase (or decrease) in percentage of the total wealth after taking part at one round of the game.
The output is then the average efficiency over the period of time between the 2 prints.

\end{itemize}


\textbf{In the Fcts.h header:}

\subsection*{updatestrategy}
This function checks what the utility would be for each of the strategies that we allow to k and then samples a choice from a logit distribution.
The function used to sample the strategy is:
$$ p_{i}\left(s_{i},s_{-i}\right)=\frac{e^{\beta u_{i}\left(s_{i},s_{-i}\right)}}{\sum_{s_{i}^{'}\in S_{i}}e^{\beta u_{i}\left(s_{i}^{'},s_{-i}\right)}} $$
as eq. (1) from the paper \textit{The logit-response dynamics}.


In the second \textbf{for} cycle, when the function is computing the various utilities, if the ranking with this strategy would be the same before, instead of computing the utility by creating the groups, I compute the new utility from the old utility in this way:
$$
u_{t}=u_{t-1}+\textrm{lenghtofinterval} *w_{i}\left(Qr_{i}-1\right) 
$$
where t is the index for the current strategy and lenghtofinterval is the difference in contribution between the current and the previous strategy.
This is because
\begin{align*}
u_{t-1} &=& w\left(1-\alpha_{t-1}\right)+\sum_{j\neq i\:,\: j\in S_{i}}QO_{j}
+\alpha_{t-1}Qrw &\Rightarrow& \\
&=& w+\sum_{j\neq i\:,\: j\in S_{i}}QO_{j}+\alpha_{t-1}w\left(Qr-1\right)
 & & \\
u_{t} &=& w\left(1-\alpha_{t}\right)+\sum_{j\neq i\:,\: j\in S_{i}}QO_{j}
+\alpha_{t}Qrw &\Rightarrow& \\
u_{t} &=& u_{t-1}+\left(\alpha_{t}-\alpha_{t-1}\right)w\left(Qr-1\right) &&
\end{align*}

%\begin{align*}
%\textrm{Group efficiency:} X=\left(\sum x_{i}^{\frac{1}{\alpha}}\right)^{\alpha} \\
%\alpha = \textrm{Collaborative ability}\\
%\alpha\ll1\Rightarrow X\approxeq max\left(x_{i}\right) \\
%\alpha=1\Rightarrow \textrm{linear} \\
%x=x_{i}\Rightarrow X=n^{\alpha}x \\
%\textrm{(Lanchester-Osipov model)}
%\end{align*}

\end{document}